\documentclass{article}
% Comment the following line to NOT allow the usage of umlauts
\usepackage[utf8]{inputenc}
% Uncomment the following line to allow the usage of graphics (.png, .jpg)
%\usepackage{graphicx}

% Start the document
\begin{document}

% Create a new 1st level heading
\section{Evolution of Modern Health Care System}

\subsection{INTRODUCTION}
Twenty years ago, a lot was different in the world. Ever heard of a cell phone?
Could you have imagined computers with the Internet or TVs that connected
to them too? Politics were different as well, and this had a big influence on
healthcare. Nowadays, technology is at our fingertips 24/7. We have become
information seekers thanks to the abundance of information available. And
sadly, we have never been sicker as a population either.
Modern medicine has progressed and shifted its attention to prevention.
Efforts are underway to minimise and eradicate disease, to support general
physical and mental health, and to educate patients and their families on how
to stay safe. Because of this focus on prevention, as well as our fascination with
technology and convenience, it’s only a matter of time before


Health-care systems around the world are focusing their policy efforts in the
twenty-first century on enhancing the quality of care provided to their popula-
tions. Earlier periods, on the other hand, saw a number of seemingly unrelated
accidents and developments lead to improvements in healthcare quality. In this
research, we examine major international historical events that increased health
care quality between 1860 and 1960, including health care financing innova-
tion, care delivery innovation, and labour diversity. Many of these difficulties
continue to be faced by today’s nursing and doctor workforce around the world.







\subsection{DIGITAL HEALTH CARE SYSTEM}

It seems that everyone has a smartphone these days and with hundreds of avail-
able applications to put on these devices, why not use some for health? There
now exists a multitude of smartphone apps that will help everyday people with
symptom management for all ailments and complaints. Apps for everything,
from helping a new mother determine if her baby’s poop is appropriate, to
monitoring cancer/chemotherapy side effects now exist. With such handy tools,
it’s easier than ever for people to take control and monitor their own health,
emphasizing patient and family-centered care as well.


Digital check-ups are a new way of checking in with your practitioner to receive
primary follow-up care without having to travel far, or for something that could
be as simple as a blood pressure check or to see how a new medication is going.
The information is transferred to the physician to review and based on the
numbers shared, such as blood pressure, heart rate, or respirations, the physician
can make an informed decision about the next step of the plan




This, of course,doesn’t work for everyone or every issue, but for simple follow-up visits, it can
eliminate a lot of waiting, free up appointments for those with acute ailments,
and is more convenient for patients. If it is more convenient, it is more likely
that they will follow through, providing better overall health.

\clearpage

\subsection{AIM OF MODERN HEALTH CARE SYSTEM}

In the case of the health system, the main aim is to produce a health in the population, that is equitably distributed. However the population also expects the health system to treat people with dignity. Within this framework, health system responsiveness was given the formal definition of "the ability of the health system to meet the population's legitimate expectations regarding their interaction with the health system, apart from expectations for improvements in health or wealth". The population's legitimate expectations were defined in terms of international human rights norms and professional ethics.


WHO's work on health systems responsiveness aims to develop the technical tools to monitor and raise awareness on the issue of how people are treated and the environment in which they are treated. This area of work aims to develop the associated norms and standards for assessing responsiveness, and maintains, as part of the Equity Team, a particular focus on differences in the way people are treated that are associated with their social status


\subsection{conclusion}

The government has played a vital role in the eradication of small pox, polio,
yaws and kidney worm infestation. This all has been made possible by a robust
public health system. Even malaria, which was among the most threatening
endemic diseases in India, is now on the verge of being eliminated in the country.

The year 2014 marked a watershed moment in the history of Indian public
healthcare system. The World Health Organisation (WHO) on March 27 that
year declared India a polio-free nation — the fourth WHO region globally to
have achieved this feat after Americas (1994), the Western Pacific Region (2000)
and the European Region (2002).

The public health sector in India is undergoing a tremendous change driven
by forward looking policy initiatives, technological revolution that is fast closing
the gaps in healthcare delivery system and increasing integration of traditional
systems of medicine with modern healthcare to deliver affordable and inclusive
healthcare for all.
The World Health Organization (WHO), the United Nations system’s directing
and coordinating body for health, states that the goals of healthcare systems
are good health for citizens, responsiveness to population expectations, and fair
means of paying operations. Progress toward them is contingent on how well sys-
tems perform four critical functions: health-care delivery, resource production,
financing, and stewardship. Quality, efficiency, acceptability, and equity are
some of the other factors to consider when evaluating health systems. They’re
also known as ”the five C’s” in the United States: Cost, Coverage, Consistency,
Complexity, and Chronic Illness



% Uncomment the following two lines if you want to have a bibliography
%\bibliographystyle{alpha}
%\bibliography{document}

\end{document}
